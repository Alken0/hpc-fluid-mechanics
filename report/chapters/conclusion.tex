\chapter{Conclusion}\label{ch:conclusion}
In this project several experiments were successfully run.
This includes the \textit{Shear Wave Decay} with 2 different initial conditions.
In addition, the correlation between viscosity and $\omega$ was measured.

Furthermore, the \textit{Couette Flow} and \textit{Poiseulle Flow} were run.
Both experiments showcased, how fluid behaves with different outer boundaries.
While the \textit{Couette Flow} showed the behavior of a sliding lit with static ground in \cref{sec:couette-flow}, the \textit{Poiseulle Flow} introduces pressure to bring movement into a fluid in a pipe in \cref{sec:poiseuille-flow}.
Both experiments showed especially the effects of friction to the movement of the fluid.

Finall the \textit{Sliding Lit} experiment combined elements of the previous experiments to simulate a fluid in a box with a sliding lit.
The project was further parallelized as shown in \cref{sec:parallelization}.
The parallelization showed a decreasing effectiveness with the number of processes in \cref{sec:sliding-lit}, which is explainable by the communication overhead.
\newline

Overall the experiments were quite successfull, there were only minor problems.
One of them was the datatype of the probability density function.
The type was in the beginning only float32, which led to unstable behavior in later steps in the experiment.
\newline

The project in general is written in a very extensable manner with a focus on clean code.
Therefore it could be extended in the future.
Another experiment, that could be conducted, would be the introduction of an obstacle in the center of the simulation.
This leads to different streams around the obstacle, depending on its shape.


- tODOs
- CHATGPT erwähene!