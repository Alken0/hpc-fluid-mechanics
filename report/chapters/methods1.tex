\chapter{Methods}

The Boltzmann Transport Equation (BTE) serves as a fundamental framework for describing the temporal evolution of the particle probability density function, denoted as \(f(\mathbf{x}, \mathbf{v}, t)\).
This function captures essential information about the distribution of particles in a system, taking into account their microscopic velocities \(\mathbf{v}\) and positions \(\mathbf{x}\).
To achieve equilibrium, the BTE guides the relaxation of the particle distribution towards the renowned Maxwell velocity distribution function, as first established by Huang \cite{huang1963statistical}.
\newline

In this pursuit, a pivotal concept arises—namely, the approximation of the relaxation process from the initial distribution \(f\) towards the equilibrium distribution \(f^{\mathrm{eq}}\).
This approximation, initially proposed by Bhatnagar, Gross, and Krook \cite{bhatnagar1954model}, elucidates the manner in which the system converges towards a state of balance, as the dynamics of \(f\) gradually approach those of \(f^{\mathrm{eq}}\).
By characterizing this relaxation phenomenon, we gain valuable insights into the underlying mechanisms governing the equilibrium attainment in systems governed by the BTE.