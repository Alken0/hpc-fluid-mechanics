\chapter{Implementation}


\section{Setup}
This project was developed using pyenv and pip.
Pyenv was selected due to its ability to create a virtual Python environment while still utilizing pip as the native package manager, distinguishing it from alternatives like anaconda.
\newline

The project's requirements are outlined in the requirements.txt file, and the desired Python version is specified in the .python-version file, generated by pyenv.
To install the requirements, execute the following commands in the project's top directory within a Bash environment:

\begin{center}
    \begin{lstlisting}[language=bash]
#!/bin/bash
pyenv install 3.11
pyenv local 3.11
source venv/bin/activate
pip install -r requirements.txt
    \end{lstlisting}
\end{center}


\section{Code Structure}
The project consists of three primary folders: src/shared, src/experiments, and tests.
The tests folder contains code dedicated to programmatic validation of the implementations, primarily comprising unit tests.
The src/shared folder contains code that is shared among all experiments conducted in the project.
This includes implementations for streaming and collision in the lib file, among others.
Lastly, the src/experiments folder contains experiment-specific code, including plotting functionalities tailored to each experiment.
